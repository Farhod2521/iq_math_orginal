\documentclass{article}
\usepackage[utf8]{inputenc}
\usepackage[T1]{fontenc}
\usepackage[russian, english, uzbek]{babel}
\usepackage{amsmath} % For math symbols
\usepackage{array} % For better table formatting
\usepackage{booktabs} % For professional tables

\begin{document}

\begin{table}[h]
\centering
\begin{tabular}{|l|l|l|l|l|}
\hline
TOPIC\textbackslash{}_NAME: Kasrlarni bo‘lish. & TOPIC\textbackslash{}_NAME: Kasrlarni bo‘lish. & TOPIC\textbackslash{}_NAME: Kasrlarni bo‘lish. &  &  \\
\hline
level & question\textbackslash{}_text\textbackslash{}_uz & correct\textbackslash{}_text\textbackslash{}_answer\textbackslash{}_uz & question\textbackslash{}_text\textbackslash{}_ru & correct\textbackslash{}_text\textbackslash{}_answer\textbackslash{}_ru \\
1 & sonining teskari sonini toping. &  & Укажите обратное числу: &  \\
1 & Bo‘ling. &  & Разделите: &  \\
1 & ni  ga bo‘ling. &  & Разделите  на  . &  \\
1 & Ali do‘kondan 4 dona qalam sotib oldi. Qalamlarning barchasi bir xil narxda edi. Agar Ali qalam uchun  dollar to‘lagan bo‘lsa, bir dona qalam narxi qancha bo‘lgan? &  & Али купил в магазине 4 карандаша. Все карандаши стоили одинаково. Если бы Али заплатил за карандаш   доллара, какова была бы цена одного карандаша? & цента \\
1 & ni  ga bo‘ling. & 1 & Разделите  на  . & 1 \\
1 & sonining teskari sonini toping. &  & Укажите обратное числу: &  \\
7 & Bir laganda  kilogram shirinlik bor edi. Shirinlik har biri  kilogramm bo‘lgan bo‘lakchalarga bo‘lindi. Shirinlik jami nechta bo‘lakka bo‘lindi? & 4 & В одном лотке было  килограмма сладостей. Десерт был порезан на куски по    килограммов каждый. На сколько частей был разделен десерт? & 4 \\
8 & ni  ga bo‘ling. &  & Разделите  на  . &  \\
9 & ni  ga bo‘ling. &  & Разделите  на   . &  \\
10 & sonining teskari sonini toping. &  & Укажите обратное числу: &  \\
11 & ni  ga bo‘ling. &  & Разделите  на   . &  \\
12 & Velosipedchining tezligi  km/soat. U  19 km ni necha soatda bosib o‘tadi? &  & Скорость велосипедиста  км/ч. За сколько часов он проедет 19 км? &  \\
13 & Bo‘ling. 15:25 &  & Разделите: 15:25 &  \\
14 & Bir shisha  litr sutga ega. Har bir stakan  litrga teng bo‘lsa, shishadagi sut bilan nechta stakan to‘ldirsa bo‘ladi? & 2 & Бутылка вмещает   литра молока. Сколько стаканов можно наполнить бутылкой молока, если каждый стакан  литра? & 2 \\
15 & vasonlarining ayirmasiga teskari son yozing. &  & Укажите обратное числу: произведение . &  \\
\hline
\end{tabular}
\end{table}

\end{document}
